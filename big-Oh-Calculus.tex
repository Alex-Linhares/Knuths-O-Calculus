% This is the original form of a letter that I submitted to the
% Notices of the American Mathematical Society in March, 1998.
% I had to shorten it for publication, because they generally limit
% letters to less than one page.  - Don Knuth

% It uses plain TeX conventions: Just say "tex ocalc" and print the result.

\magnification =\magstephalf
\def\AW{Addison\kern.1em--Wesley}
\def\adx#1:#2\par{\par\halign{\hskip #1##\hfill\cr #2}\par}
\def\disleft#1:#2:#3\par{\par\hangindent#1\noindent
	 \hbox to #1{#2 \hfill \hskip .1em}\ignorespaces#3\par}
\parskip5pt
\parindent0pt

\adx0pt:
Professor Anthony W. Knapp\cr
P O Box 333\cr
East Setauket, NY 11733\cr

\bigskip
Dear editor,

I am pleased to see so much serious attention being given to improvements
in the way calculus has traditionally been taught, but I'm surprised that
nobody has been discussing the kinds of changes that I personally believe
would be most valuable.  If I~were responsible for teaching calculus to
college undergraduates and advanced high school students today, and if
I~had the opportunity to deviate from the existing textbooks, I~would
certainly make major changes by emphasizing several notational improvements
that advanced mathematicians have been using for more than a hundred years.

The most important of these changes would be to introduce the $O$~notation
and related ideas at an early stage.  This notation, first used by Bachmann
in 1894 and later popularized by Landau, has the great virtue that it makes
calculations simpler, so it simplifies many parts of the subject, yet it is
highly intuitive and easily learned.  The key idea is to be able to deal
with quantities that are only partly specified, and to use them in the
midst of formulas.

I would begin my ideal calculus course by introducing a simpler
``$A$~notation,'' which means ``absolutely at most.''  For example,
$A(2)$~stands for a quantity whose absolute value is less than or equal
to~2.  This notation has a natural connection with decimal numbers:  Saying
that $\pi$ is approximately 3.14 is equivalent to saying that
$\pi=3.14+A(.005)$.  Students will easily discover how to calculate
with~$A$:
$$\eqalign{&10^{A(2)}=A(100)\,;\cr
&\bigl(3.14+A(.005)\bigr)\bigl(1+A(0.01)\bigr)\cr
&\quad=3.14+A(.005)+A(0.0314)+A(.00005)\cr
&\quad=3.14+A(0.3645)=3.14+A(.04)\,.\cr}$$
I would of course explain that the equality sign is not symmetric with
respect to such notations; we have $3=A(5)$ and $4=A(5)$ but not
$3=4$, nor can we say that $A(5)=4$.  We can, however, say that $A(0)=0$.
As de~Bruijn points out in [1,~\S 1.2], mathematicians customarily use
the $=$~sign as they use the word ``is'' in English: Aristotle is a man,
but a man isn't necessarily Aristotle.

The $A$ notation applies to variable quantities as well as to constant
ones.  For example,
$$\vcenter{\halign{\hfil$#\;$&$#$\hfil\quad&#\hfil\cr
\sin x&=A(1)\,;\cr
x&=A(x)\,;\cr
A(x)&=xA(1)\,;\cr
A(x)+A(y)&=A(x+y)&if $x\geq 0$ and $y\geq 0\,;$\cr
\bigl(1+A(t)\bigr){}^2&=1+3A(t)&if $t=A(1)\,.$\cr}}$$

Once students have caught on to the idea of $A$~notation, they are ready
for $O$~notation, which is even less specific.  In its simplest form,
$O(x)$ stands for something that is $CA(x)$ for some constant~$C$, but we
don't say what $C$~is.  We also define side conditions on the variables
that appear in the formulas.  For example, if $n$ is a positive integer we can
say that any quadratic polynomial in $n$ is $O(n^2)$.  If $n$ is sufficiently
large, we can deduce that
$$\eqalign{&\bigl(n+O(\sqrt{n}\,)\bigr)\bigl(\ln n+\gamma+O(1/n)\bigr)\cr
&\quad=n\ln n+\gamma n+O(1)\cr
&\qquad\null+O(\sqrt{n}\ln n)+O(\sqrt{n}\,)+O(1/\sqrt{n}\,)\cr
&\quad=n\ln n+\gamma n+O(\sqrt{n}\ln n)\,.\cr}$$

I would define the derivative by first defining what might be called a
``strong derivative'':  The function~$f$ has a strong derivative $f'(x)$ at
point~$x$ if
$$f(x+\epsilon)=f(x)+f'(x)\epsilon+O(\epsilon^2)$$
whenever $\epsilon$ is sufficiently small.  The vast majority of all functions
that arise in practical work have strong derivatives, so I~believe this
definition best captures the intuition I~want students to have about
derivatives.  We see immediately, for example, that if $f(x)=x^2$ we have
$$(x+\epsilon)^2=x^2+2x\epsilon+\epsilon^2\,,$$
so the derivative of $x^2$ is $2x$.  And if the derivative of $x^n$ is
$d_n(x)$, we have
$$\eqalign{(x+\epsilon)^{n+1}&=(x+\epsilon)\bigl(x^n+d_n(x)\epsilon%
+O(\epsilon^2)\bigr)\cr
&=x^{n+1}+\bigl(xd_n(x)+x^n\bigr)\epsilon+O(\epsilon^2)\,;\cr}$$
hence the derivative of $x^{n+1}$ is $xd_n(x)+x^n$ and we find by induction
that $d_n(x)=nx^{n-1}$.  Similarly if $f$ and~$g$ have strong derivatives
$f'(x)$ and $g'(x)$, we readily find
$$f(x+\epsilon)g(x+\epsilon)=f(x)g(x)+\bigl(f'(x)g(x)+f(x)g'(x)\bigr)\epsilon
+O(\epsilon^2)$$
and this gives the strong derivative of the product.  The chain rule
$$f\bigl(g(x+\epsilon)\bigr)=f\bigl(g(x)\bigr)+f'\bigl(g(x)\bigr)g'(x)\epsilon
+O(\epsilon^2)$$
also follows when $f$ has a strong derivative at point $g(x)$ and $g$ has a
strong derivative at~$x$.

Once it is known that integration is the inverse of differentiation and
related to the area under a curve, we can observe, for example, that if $f$
and~$f'$ both have strong derivatives at~$x$, then
$$\eqalign{f(x+\epsilon)-f(x)&=\int_0^{\epsilon}f'(x+t)\,dt\cr
\noalign{\smallskip}
&=\int_0^{\epsilon}\bigl(f'(x)+f''(x)\,t+O(t^2)\bigr)\,dt\cr
\noalign{\smallskip}
&=f'(x)\epsilon+f''(x)\epsilon^2\!/2+O(\epsilon^3)\,.\cr}$$

I'm sure it would be a pleasure for both students and teacher if calculus
were taught in this way.  The extra time needed to introduce $O$~notation
is amply repaid by the simplifications that occur later.  In fact, there
probably will be time to introduce the ``$o$~notation,'' which is
equivalent to the taking of limits, and to give the general definition
of a not-necessarily-strong derivative:
$$f(x+\epsilon)=f(x)+f'(x)\epsilon+o(\epsilon)\,.$$
The function $f$ is continuous at $x$ if
$$f(x+\epsilon)=f(x)+o(1)\,;$$
and so on.  But I would not mind leaving a full exploration of such things
to a more advanced course, when it will easily be picked up by anyone who
has learned the basics with~$O$ alone.  Indeed, I~have not needed to use
``$o$'' in 2200 pages of {\sl The Art of Computer Programming}, although
many techniques of advanced calculus are applied throughout those books to
a great variety of problems.

Students will be motivated to use $O$ notation for two important
reasons. First, it significantly simplifies calculations because it allows
us to be sloppy---but in a satisfactorily controlled way.  Second, it
appears in the power series calculations of symbolic algebra systems like
{\sl Maple\/} and {\sl Mathematica}, which today's students will surely be
using.

For more than 20 years I have dreamed of writing a calculus text entitled
{\sl O Calculus}, in which the subject would be taught along the lines
sketched above.  More pressing projects, such as the development of the
\TeX\ system, have made that impossible, although I~did try to write a good
introduction to $O$~notation for post-calculus students in [2,~Chapter~9].
Perhaps my ideas are preposterous, but I'm hoping that this letter will
catch the attention of people who are much more capable than~I of writing
calculus texts for the new millennium.  And I~hope that some of these
now-classical ideas will prove to be at least half as fruitful for
students of the next generation as they have been for me.

\adx 150pt:
Sincerely,\cr
Donald E. Knuth\cr
Professor\cr

\adx0pt:
DEK/pw\cr

\medskip

\disleft 20pt:[1]:
N. G. de Bruijn, {\sl Asymptotic Methods in Analysis\/} (Amsterdam:
North-Holland, 1958).
\smallskip
\disleft 20pt:[2]:
R. L. Graham, D. E. Knuth, and O. Patashnik, {\sl Concrete Mathematics\/}
(Reading, Mass.: Addison\kern.1em--Wesley, 1989).

\bye
